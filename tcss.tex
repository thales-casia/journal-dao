\documentclass[lettersize,journal]{IEEEtran}
\usepackage{amsmath,amsfonts}
\usepackage{algorithmic}
\usepackage{algorithm}
\usepackage{array}
\usepackage[caption=false,font=normalsize,labelfont=sf,textfont=sf]{subfig}
\usepackage{textcomp}
\usepackage{stfloats}
\usepackage{url}
\usepackage{verbatim}
\usepackage{graphicx}
\usepackage{cite}
\hyphenation{op-tical net-works semi-conduc-tor IEEE-Xplore}
% updated with editorial comments 12/2/2022

\begin{document}

\title{Journal DAO: A New Framework for DeSci in Web 3.0}

\author{Tai Jiang,~\IEEEmembership{Follow,~IEEE,}
        % <-this % stops a space
\thanks{Identify applicable funding agency here. If none, delete this.}% <-this % stops a space
\thanks{Manuscript received February 15, 2023; revised February 15, 2023.}}

% The paper headers
\markboth{Journal of \LaTeX\ Class Files,~Vol.~14, No.~8, February~2023}%
{Shell \MakeLowercase{\textit{et al.}}: A Sample Article Using IEEEtran.cls for IEEE Journals}

% \IEEEpubid{0000--0000/00\$00.00~\copyright~2021 IEEE}
% Remember, if you use this you must call \IEEEpubidadjcol in the second
% column for its text to clear the IEEEpubid mark.

\maketitle

\begin{abstract}
The advent of Web3.0 signifies a notable evolution in the Internet landscape, emphasizing decentralization, openness, and user control over data. This paper begins by delineating the key characteristics of Web3.0, including a decentralized network architecture, the application of smart contracts, and the protection of data privacy.Against the backdrop of blockchain technology, leveraging the immutability and transparency of blockchain, the paper discusses how it can facilitate the authenticity, security, and traceability of paper data, providing a new perspective on ownership and governance.Lastly, the paper proposes the structure of a decentralized autonomous organization (DAO) and discusses its potential applications in resolving ownership and governance issues. Through the DAO framework, a more just, transparent, and open organizational structure can be realized, bringing new governance models to the academic research domain. This structure underscores more equitable stakeholder participation, fostering the collaborative development of the academic community.By comprehensively examining Web3.0, blockchain applications, ownership issues, and the structure of DAO, this paper aims to provide insights for future research directions and innovative applications in the academic and technological communities.
%Web3.0的兴起标志着互联网的进一步演进,强调了去中心化、开放性和用户数据的掌控权。本文首先介绍了Web3.0的主要特点,包括去中心化的网络架构、智能合约的应用、数据隐私的保护等方面。在区块链技术的背景下,通过区块链的不可篡改性和透明性,可以实现论文数据的真实、安全和可追溯,为所有权和治理提供了新的观点。最后,本文提出了去中心化组织(DAO)的结构,探讨了其在解决所有权和治理方面的潜在应用。通过DAO的框架,可以实现更加公正、透明和开放的组织结构,为学术研究领域带来新的治理模式。这一结构强调了利益相关者之间更平等的参与,促进了学术社区的共同发展。通过对Web3.0、区块链应用、所有权问题以及去中心化组织结构的综合探讨,本文旨在为学术界和科技社区提供对于未来研究方向和创新应用的启示。
\end{abstract}


\begin{IEEEkeywords}
DAO, smart contract, decentralized autonomous organizations, decentralized funding, decentralized science, DeSci, parallel DeSci, Web3
\end{IEEEkeywords}

\section{Introduction}
%分段写清楚期刊的变化,引出现在需要做的主要内容

\IEEEPARstart{A}{cademic} publication methods have undergone significant historical transformations, reflecting the evolution of technology, society, and culture.Traditional scholarly journals, rooted in print formats and centralized editorial processes, have long been the primary medium for disseminating research findings \cite{febvre1997coming}. However, the digital age has brought about profound changes, challenging the conventional models and paving the way for innovative solutions \cite{meadows1997communicating}. This paper delves into the historical trajectory of academic journals, highlighting key milestones and contextualizing the current paradigm shift towards blockchain technology.
%学术论文的发表方式在历史上发生了显著的变迁,反映了技术、社会和文化的演进。传统的学术期刊,根植于印刷格式和中心化编辑流程,长期以来一直是传播研究成果的主要媒介。然而,数字时代带来了深刻的变革,挑战传统模式,为创新性解决方案铺平了道路。本文深入探讨学术期刊的历史轨迹,突显关键里程碑,并详细阐述当前朝着区块链技术过渡的背景。

The early days of academic publishing were marked by the establishment of prestigious print journals, serving as repositories of scholarly knowledge. These publications played a crucial role in shaping academic discourse, but their centralized nature often led to challenges such as delayed dissemination and limited accessibility. With the advent of the internet, electronic journals emerged, offering faster dissemination and broader reach, yet grappling with issues of copyright, authenticity, and peer review integrity.In recent years, the emergence of blockchain technology has presented a transformative opportunity for the academic publishing landscape. Blockchain, with its decentralized and immutable ledger, addresses several longstanding challenges \cite{swan2015blockchain}. The transition to blockchain-based academic journals holds the promise of enhancing transparency, traceability, and security in scholarly communication.
%学术出版的早期以建立声望卓著的印刷期刊为特征,这些期刊作为学术知识的存储库发挥着至关重要的作用。这些出版物在塑造学术话语方面发挥了关键作用,但由于其中心化的特性,往往面临着信息传播延迟和获取受限等挑战。随着互联网的出现,电子期刊应运而生,提供了更快的传播速度和更广泛的覆盖面,但同时也面临着版权、真实性和同行评审完整性等问题。
%近年来,区块链技术的出现为学术出版领域带来了深刻的变革机遇。区块链凭借其去中心化和不可篡改的账本,解决了多项长期存在的问题。向基于区块链的学术期刊过渡有望提升学术交流的透明性、可追溯性和安全性。




This paper sheds light on the dynamic evolution of research publication methods, continually adapting to new technologies and societal trends. The future of academic publishing is expected to witness further transformations, including expanded open access, increased peer review transparency, enhanced international collaboration, and interdisciplinary research integration. These trends are poised to shape the future of research dissemination.
%这些趋势反映了学术论文发表方式的不断演化,以适应新的技术和社会趋势。未来,学术出版领域可能会继续面临新的变革,例如更广泛的开放获取、更透明的同行评审、更多的国际合作和跨学科研究等。这些趋势有望继续塑造学术发表的未来。

In the realm of academic journals, the integration of blockchain technology, smart contracts, and Decentralized Autonomous Organizations (DAO) represents a transformative frontier. Blockchain ensures the immutability and transparency of data, addressing issues of data integrity and reliability. Smart contracts, operating on the blockchain, automate and enforce the rules governing various processes, such as article submissions, peer reviews, and fund distribution. DAOs, as a governance structure, provide a decentralized decision-making framework, fostering inclusivity and community engagement.The synergistic application of blockchain, smart contracts, and DAOs can enhance the efficiency and integrity of academic journals. It introduces a trustless and transparent environment, reducing the need for intermediaries and ensuring that the entire process, from article submission to fund distribution, is executed with a high level of security and accountability. This paradigm shift has the potential to redefine the landscape of academic publishing, fostering a more collaborative, secure, and accessible platform for scholarly communication.
%在学术期刊领域,区块链技术、智能合约和去中心化自治组织(DAO)的整合代表了一种变革性的前沿。区块链确保了数据的不可篡改性和透明性,解决了数据完整性和可靠性的问题。智能合约在区块链上运行,自动执行并执行规范各种流程的规则,如文章提交、同行评审和资金分配。DAO作为一种治理结构,提供了一个去中心化的决策框架,促进包容性和社区参与。
%区块链、智能合约和DAO的协同应用可以增强学术期刊的效率和完整性。它引入了一个无需信任且透明的环境,减少了中介的需求,确保从文章提交到资金分配的整个过程都具有高水平的安全性和问责制。这种范式转变有望重新定义学术出版的格局,促进更具合作性、安全性和可访问性的学术交流平台的发展。

\begin{figure}[h]
  \centering
  \includegraphics[width=\linewidth]{assets/dao.png}
  \caption{Application of DAO in Article and Journal}
  \label{fig:applicationofdao}
\end{figure}


Application of DAO in Article and Journal as figure \ref{fig:applicationofdao} shows:

\begin{itemize}
  \item \textbf{Transparent Peer Review and Publishing Process:} 
  
  Blockchain can be used to create a transparent academic publishing platform, ensuring transparency throughout the peer-review and publishing processes.\cite{nakamoto2008bitcoin} Smart contracts can manage review, publication, and payment procedures, ensuring traceability and fairness.
  %透明的学术出版: 区块链可以用于创建透明的学术出版平台,确保审稿和出版过程的透明性。智能合约可以管理审稿、出版和付款流程,确保可追溯和公平。

  \item \textbf{Protection of Intellectual Property:}
  
  Blockchain and smart contracts can safeguard authors' intellectual property, ensuring their works are not copied or distributed without permission.
  %知识产权保护: 区块链和智能合约可用于保护作者的知识产权,确保其作品不会未经许可就被复制或传播。
  
  \item \textbf{Identity Verification and Reputation Building:} 
  
  Blockchain can be employed to establish scholars' identities and reputations \cite{radziwill2018blockchain}. Smart contracts can automate the validation of scholars' achievements, storing them on the blockchain.
  %身份验证和声誉建立: 区块链可用于建立学者的身份和声誉。智能合约可以自动化验证学者的成就,并将其存储在区块链上。
  
  \item \textbf{Data Sharing and Collaboration:} 
  
  Blockchain and smart contracts can facilitate data sharing and collaboration among scholars, ensuring data integrity and traceability \cite{praitheeshan2019security}.
  %数据共享和合作: 区块链和智能合约可用于促进学者之间的数据共享和合作,确保数据的完整性和来源可追溯。
  
  \item \textbf{Finance and Incentive Mechanisms:} 
  
  DAO and cryptocurrencies can be used to support research finance and incentive mechanisms.Funds can be allocated according to token weightsn.
  %金融和激励机制: 区块链和加密货币可以用于支持学术研究的融资和激励机制。资金可以按照token为权重分配。
\end{itemize}

These application examples highlight the potential value of blockchain, smart contracts, and DAO technology in academic publishing and journal management. They enhance transparency, protect intellectual property, verify identity, automate processes, and encourage collaboration. As these technologies continue to evolve, they hold promise for further innovation and efficiency in academia.
%这些应用示例突显了区块链、智能合约和DAO技术在学术出版和杂志管理方面的潜在应用,提高了透明度、知识产权保护、身份验证和自动化审稿流程。随着这些技术的不断发展,它们将有望为学术界提供更多的创新和效率。

In the traditional landscape of Web 2.0, the ownership of articles remains with the creators, yet the entirety of the associated data is ensconced within the databases of publishers, shrouded in a lack of transparency. This opaqueness extends to financial allocations, where, even if publishers express willingness to distribute funds, the absence of intermediary oversight poses challenges to ensuring fairness and accountability in the process. This paper delves into the transformative potential of Web 3.0 and blockchain technology within the DAO framework to rectify these issues, ushering in a new era of decentralized, transparent, and equitable financial distribution for content creators.
%在传统的Web 2.0领域,文章的所有权虽然归属于创作者,但所有相关数据都存储在由出版商的数据库中,缺乏透明度。即使出版商愿意分配资金,由于缺乏中介监管,这一过程也存在公正性和责任问题。本文深入探讨了Web 3.0和区块链技术在DAO框架内的转变潜力,以解决这些问题,引领创作者进入一个新时代,其中财务分配更为去中心化、透明和公平。

This paper underscores the transformative impact of blockchain technology on authorial ownership within the DAO framework. By securing and amplifying authorship rights, the paper explores how this paradigm shift facilitates a fair and transparent allocation of finance. Through the inherent decentralization and smart contract capabilities of DAOs, the financial ecosystem becomes a dynamic and equitable space, fostering an environment where creators are duly recognized and rewarded for their contributions.
%本文强调了区块链技术在DAO框架下对作者所有权的重大影响。通过确保并增强作者的权利,本文探讨了这种范式转变如何促进对财务资源的公正和透明分配。通过DAO的固有去中心化和智能合约能力,财务生态系统变得动态而公正,营造了一个环境,让创作者得到应有的认可,并因其贡献而得到公正的回报。

\section{Journal Data Ownership in the System of Web3.0}
% 相关工作,2到3个点,现在已经存在的

This chapter delves into the intricate matter of research data ownership. In the realm of academic research, the question of who possesses, controls, and manages research data is of paramount importance, involving researchers, academic institutions, publishers, and various stakeholders within society. We also investigate existing data-sharing models and open-access policies and their potential effects on the academic community and knowledge innovation. By delving into the issue of research data ownership, we gain a deeper understanding of the challenges and opportunities in today's academic environment, offering new perspectives for future research and collaboration.
%本章将深入探讨论文数据的所有权问题。在学术研究领域,谁拥有、控制和管理研究数据是一个关键议题,涉及到研究人员、学术机构、出版商和社会的各个利益相关方。我们还将研究现有的数据共享模式和开放获取政策,以及它们对学术社区和知识创新的潜在影响。通过深入研究论文数据的所有权问题,我们将更好地理解当前学术环境中的挑战和机遇,为未来的研究和合作提供新的思考。

\begin{figure}[h]
  \centering
  \includegraphics[width=2in]{assets/journalsite.png}
  \caption{Article in Database}
  \label{fig:journalsite}
\end{figure}

From a physical perspective like Figure \ref{fig:journalsite}, the data of articles is stored in the database of the journal website. When regular users access the journal website, they can browse and download articles of interest. The interaction between users and the journal website typically involves the following steps.
%从物理层出发,期刊网站的数据库是文章数据的关键存储点。这些数据库通常位于服务器上,这些服务器可以分布在数据中心或云服务提供商的设施中。当普通用户访问期刊网站时,他们可以通过网络连接访问这些数据库以浏览和下载感兴趣的论文。


\begin{figure}[h]
  \centering
  \includegraphics[width=3in]{assets/journalchain.png}
  \caption{Article in Blockchain}
  \label{fig:journalchain}
\end{figure}

When an article is uploaded to a blockchain like Figure \ref{fig:journalchain}, its content, timestamp, and relevant metadata are all recorded on the blockchain. This means that anyone can verify the existence, content, and timestamp of the article. This provides a high level of assurance for the immutability and transparency of documents, particularly with potential significance in research, intellectual property protection, and copyright. Uploading articles to the blockchain also enables decentralized data storage, reducing reliance on centralized institutions. This offers a more open and trustworthy means of data sharing for the academic community and other domains.Uploading an article to a blockchain, as compared to storing it in a traditional database, provides the author with a clear and objective ownership of the article. In a traditional database, the ownership of the data and the integrity of the database are controlled by the entity or organization managing the database. Authors and other stakeholders may not have direct control or visibility into the ownership and usage of the data.
%%当一篇文章被上链到区块链中,它的内容、时间戳以及相关的元数据都会被记录在区块链上。这意味着任何人都可以验证这篇文章的存在、内容和时间。这为文献的不可篡改性和透明性提供了极高的保障,尤其在科研、知识产权保护和版权方面具有潜在的重要应用。上链文章还可以实现去中心化的数据存储,减少对中央化机构的依赖。这为学术界和其他领域提供了更开放和可信的数据共享方式。将文章上传至区块链,与将其存储在传统数据库中相比,可以让作者更明确且客观地拥有文章。在传统数据库中,数据的所有权和数据库的完整性由管理数据库的实体或组织控制。作者和其他利益相关方可能无法直接控制或了解数据的所有权和使用情况。

On the other hand, when an article is uploaded to a blockchain, the author can have greater confidence in their ownership and control over the article. The blockchain's decentralized and immutable nature ensures that the ownership records are transparent, tamper-resistant, and not under the sole control of a centralized authority. This empowers authors to have a direct and verifiable claim to their work, which can be particularly important for intellectual property protection, copyright, and ensuring that the author's rights are respected.
%另一方面,当文章上传至区块链时,作者可以更有信心地拥有和控制文章。区块链的去中心化和不可篡改性确保了所有权记录的透明性、防篡改性,并不受中心化权威的独立控制。这使作者能够直接并且可验证地主张他们的作品,这对于知识产权保护、版权以及确保作者的权利得到尊重尤为重要。



\textbf{Compared with traditional journals, articles on the chain belong entirely to the author.}
%相比于传统期刊,链上的文章完全属于作者。


\paragraph{Traditional Databases}
%传统数据库:
\begin{itemize}
  \item Traditional databases are typically controlled by a central entity or organization, with database administrators responsible for management and access control. This centralization may result in less transparent ownership.
%传统数据库通常由中央机构或组织控制,数据库管理员负责管理和授予访问权限。这可能导致数据的所有权不够透明。

  \item Access to and modification of data in traditional databases often depend on access controls set by database administrators. This can lead to disputes or lack of transparency regarding data access.
%数据的访问和修改通常依赖于数据库管理员设置的访问控制。这可能导致对数据访问权限的争议或不透明性。

  \item Data in traditional databases can be relatively easily modified or deleted. This may raise concerns about data security and integrity, especially in cases where intellectual property protection is crucial.
%传统数据库中的数据可以相对容易地被修改或删除。这可能引发数据安全性和完整性的问题,尤其是在需要保护知识产权时。

  \item Ownership and control of data are typically centralized with the database administrator. This centralization introduces a single point of control over data usage, increasing the risk of misuse or improper handling.
%数据的所有权和控制通常集中在数据库的管理者手中。这可能导致对数据使用的单一控制点,增加了数据被滥用或不当处理的风险。
\end{itemize}

\paragraph{Blockchain}
%区块链:

\begin{itemize}
  \item Every transaction on the blockchain has a clear timestamp, documenting the transfer of ownership. This provides authors with a transparent, immutable record of ownership. Authors can trace ownership back to each stage of the data's lifecycle.
%区块链上的每个交易都有明确的时间戳,记录了数据的所有权转移。这为作者提供了清晰、透明且不可篡改的所有权记录。作者可以追溯到每个阶段数据的所有者。

  \item Blockchain utilizes smart contracts to define and enforce data access permissions. This allows dynamic adjustments of data access rights based on different conditions, such as paid access or specific usage licenses.
%区块链可以使用智能合约来定义和执行数据的使用权限。这使得在不同的条件下,如付费访问或特定使用许可下,可以动态地调整数据的访问权限。

  \item Blockchain is decentralized, with data stored across multiple nodes in the network. This ensures that no single central entity can unilaterally control ownership, enhancing data security and tamper resistance.
%区块链是去中心化的,数据存储在网络的多个节点上。这意味着没有单一的中央机构能够单方面控制数据的所有权,增加了数据的安全性和防篡改性。

  \item Once data is written to the blockchain, it is nearly impossible to modify or delete. This ensures the immutability of data, providing robust protection for the author's rights.
% 一旦数据被写入区块链,几乎不可能对其进行修改或删除。这确保了数据的不可篡改性,作者的权利得到了更强有力的保护。
\end{itemize}
In summary, blockchain offers a more transparent, immutable, and decentralized ownership mechanism, providing stronger protection for authors' intellectual property and data rights. This is particularly advantageous in scenarios where emphasis is placed on data security, traceability, and transparency.using a blockchain for article storage offers authors a more objective and transparent means of claiming ownership of their work compared to traditional database systems.
%综合而言,区块链提供了更为透明、不可篡改、去中心化的拥有权机制,更好地保护了作者的知识产权和数据权益。这对于需要强调数据的安全性、可追溯性和透明性的应用场景非常有利。使用区块链进行文章存储相对于传统数据库系统,为作者提供了更客观和透明的方式来主张他们的作品所有权

\section{Author Finance from Journal}
%3 架构 我的内容 journal DAO,笼统一些的架构

In the evolution of electronic journals, websites have become the primary medium for disseminating research papers. Although the content of users' papers remains the intellectual property of the authors, the wealth generated by journals through these papers often belongs predominantly to the journals rather than the authors.In our hypothetical scenario, we contemplate a shift in this paradigm, envisioning a system where a certain proportion of the generated wealth is allocated back to the authors. Taking paid downloads as a simple example, this mechanism aims to provide authors with a more direct economic incentive. Such a transformation not only has the potential to enhance authors' motivation and creativity but also holds the promise of establishing a more equitable wealth distribution system. This envisioned change could contribute to fostering a sustainable and mutually beneficial development model in the realm of electronic journals, addressing the balance of interests between authors and journals more effectively.
% %在电子期刊的发展历程中,论文的载体逐渐从传统的印刷版本过渡到了数字化的网站形式。尽管用户的论文内容仍然属于作者的创作成果,然而,由于现行模式下,杂志通过论文创造的财富往往主要归属于杂志自身,而非作者。在当前设想下,我们可以假设一种改进模式,即按照一定比例将创造的财富回馈给作者。以付费下载为例,这一机制可以为作者提供更直接的经济回报。这种变革不仅有助于激发作者的积极性和创作动力,同时也有望促进更公正的财富分配体系的建立。这样的改变有望在电子期刊领域引入更加可持续和互惠的发展模式,从而更好地满足作者和杂志之间的利益平衡。


Web2, also known as the social web, refers to the current state of the internet that we use today, which is primarily focused on social media, e-commerce, and other web-based applications that allow users to interact with each other and with content in various ways. In Web2, payment systems are typically centralized, meaning that they are controlled by a single entity or organization. 
%Web2,也称为社交网络,指的是我们今天使用的互联网的现状,主要集中在社交媒体、电子商务和其他基于网络的应用程序,这些应用程序允许用户相互交互以及与其他人进行交互。内容以多种方式呈现。在 Web2 中,支付系统通常是集中式的,这意味着它们由单个实体或组织控制。

\begin{figure}[h]
  \includegraphics[width=3in]{assets/web2.png}
  \caption{Reader Pay for Download in Web2.0}
  \label{fig:web2}
\end{figure}

As the figure \ref{fig:web2} shows. In the context of Web 2.0, the establishment of a platform for downloading articles entails several steps. Firstly, the creation of a functional website serves as the primary interface for users. This website acts as a centralized hub, hosting a database that stores a diverse range of articles across various disciplines.When a user decides to download a specific article, a payment system is in place to facilitate the transaction. The user pays a designated fee for the download, and the platform, acting as an intermediary, manages the distribution of funds. The allocation of funds may involve a proportional distribution to the authors, and this process is typically administered by the central entity running the website.This centralized model means that all user interactions, content storage, and payment transactions occur within the controlled environment of the website. Users depend on the centralized platform to oversee and coordinate all aspects of the transactional process, creating a reliance on a single authority for the entire operation.
%在Web 2.0的背景下,建立一个文章下载平台涉及到几个步骤。首先,创建一个功能齐全的网站作为用户的主要接口。这个网站充当了一个中心枢纽,托管一个数据库,存储了各个学科领域的多样化文章。当用户决定下载特定文章时,将会有一个支付系统来促成交易。用户支付一定费用进行下载,而平台则作为中介管理资金的分配。资金的分配可能涉及按比例分配给作者,这个过程通常由运营该网站的中央实体管理。这种中心化模式意味着所有用户交互、内容存储和支付交易都发生在网站的受控环境中。用户依赖于中心化平台监督和协调整个交易过程,因此对于整个操作都存在对单一权威的依赖。



Web3, also known as the decentralized web, represents a shift toward a more open, decentralized, and secure internet that is built on blockchain technology \cite{alabdulwahhab2018web}. In Web3, payment systems are decentralized, meaning that they are not controlled by a single entity or organization \cite{cao2022decentralized}. Instead, payments are made using cryptocurrency, which is a digital asset that is secured by cryptographic techniques and operates independently of central banks and other financial institutions. Cryptocurrency payments are processed directly between users without the need for intermediaries, which can result in lower transaction fees and faster processing times.
%Web3,也称为去中心化网络,代表着向基于区块链技术构建的更加开放、去中心化和安全的互联网的转变。在 Web3 中,支付系统是去中心化的,这意味着它们不受单个实体或组织的控制。相反,支付是使用加密货币进行的,加密货币是一种由加密技术保护并独立于中央银行和其他金融机构运作的数字资产。加密货币支付直接在用户之间进行处理,无需中介机构,这可以降低交易费用并缩短处理时间。

\begin{figure}[h]
  \centering
  \includegraphics[width=3in]{assets/web3.png}
  \caption{Reader Pay for Download in Web3.0}
  \label{fig:web3}
\end{figure}



As the figure \ref{fig:web3} shows. In the realm of Web 3.0, we witness a fundamental transformation in the dissemination of academic articles. Unlike the traditional Web 2.0 model, it introduces a paradigm shift. In this innovative framework, the data entity of articles resides directly on the blockchain, with the website serving as a mere interface reflecting the blockchain data. When users make payments for downloads, the entire fund allocation process is automated through smart contracts, eliminating the need for manual intervention. This groundbreaking framework ensures complete transparency and traceability throughout the process. As users pay to download articles, funds are automatically distributed according to the rules set within the DAO framework, without any centralized oversight.In the Web 3.0 paradigm, this novel model signifies a departure from reliance on traditional intermediaries. Instead, it empowers users with the direct participation in DAO frameworks, utilizing smart contracts for automated and secure fund distribution. This shift not only achieves decentralization in the transaction process but also enhances the efficiency of academic article transactions.In the Web 3.0 environment, articles are directly uploaded to the blockchain. All operations, including payment processing and fund distribution, are seamlessly executed through smart contracts. This innovative framework ensures complete transparency and traceability throughout the entire process. When users pay to download articles, the funds are automatically allocated according to the rules established within the DAO framework, eliminating the need for any centralized oversight.
%在Web 3.0的领域中,我们对待学术文章传播方式发生了根本性的变革。与传统的Web 2.0模式不同,它引入了一种新的框架。文章数据主体直接存在于区块链上,而网站仅仅是对区块链数据的映射。在这一新的框架下,用户支付下载后,整个资金分配的过程都通过智能合约进行自动化执行,无需手动干预。这一创新性框架确保了整个过程的完全透明性和可追溯性。当用户支付下载文章时,资金会根据去中心化自治组织(DAO)框架内设立的规则自动分配,无需任何集中监管。
%在Web 3.0的范式中,这一新兴模式标志着我们不再依赖传统中介机构。相反,它赋予用户直接参与DAO框架的权力,利用智能合约实现资金的自动化和安全分配。这一转变不仅实现了交易过程的去中心化,还提高了学术文章交易的效率。
%在Web 3.0的环境中,文章直接上链。所有操作,包括支付处理和资金分配,都通过智能合约无缝执行。这一创新性框架确保了整个过程的完全透明性和可追溯性。当用户支付下载文章时,资金会根据去中心化自治组织(DAO)框架内设立的规则自动分配,无需任何集中监管。



Overall, the main difference in payment systems between Web2 and Web3 is the degree of centralization. Web2 payment systems are centralized, while Web3 payment systems are decentralized. While Web3 is still in its early stages, it has the potential to revolutionize the way we think about payments, transactions, and financial systems.In the current era of digitization, the act of anchoring a paper on the blockchain signifies the author's complete ownership of the work, opening up boundless possibilities. The introduction of blockchain technology empowers authors with more rights and limitless potential. Once a paper is inscribed on the immutable blockchain, authors not only possess intellectual property rights but also gain absolute control over their creations. This shift in ownership implies that authors can explore innovation more freely, facilitate transparent data sharing, and attain fairer returns from the wealth generated by their works. The immutability and transparency afforded by blockchain provide robust protection for the rights of paper owners, ushering in new possibilities for academic research and knowledge sharing. This profound ownership transformation elevates a paper beyond being merely a conduit for academic dissemination; it becomes a symbol of the unique wealth created by the author, sparking a profound revolution in the relationship between academia and authors.
%总体而言,Web2和Web3支付系统的主要区别在于中心化程度。 Web2支付系统是中心化的,而Web3支付系统是去中心化的。虽然 Web3 仍处于早期阶段,但它有潜力彻底改变我们对支付、交易和金融系统的看法。
%在这个时代,随着论文数字化的趋势,将论文上链意味着作者对其作品的彻底拥有。区块链技术的引入赋予了作者更多的权利和无限的可能性。一旦论文被写入不可篡改的区块链,作者不仅仅拥有了知识产权,更获得了对其作品的完全掌控。这样的所有权转变意味着作者可以更自由地探索创新、实现数据的透明共享,并在作品创造的财富中获得更为公平的回报。通过区块链的不可篡改性和透明性,论文的所有者权益得到了更强有力的保障,为学术研究和知识共享带来了新的可能性。这种彻底的拥有使得论文不再仅仅是学术传播的载体,更是作者创造的独特财富的象征,为学术界和作者之间的关系带来了深刻的变革。

  

\section{DAO to DeSci}
% 4 详细的架构, key study
% 用nft解释token

%我在写论文,下面的内容为主体写一段话,要详细,用在第四章,用英文论文的口吻:论文杂志,是用aragon创建一个journal dao,论文上区块链,发布论文时候,根据reviewer的打分发给作者一定的token,每个reviewer也会获得token。论文发布之后,有用户下载该论文,会给作者和下载论文的用户都发放token。除此之外,引用论文也会给引用者和作者发放token。按照nft的方式,一旦产生了finance,都会按照token作为全总分配finance,因为下载和引用的用户也能根据token获得finance,会激励用户下载和引用论文,完成一个完美的自治。

In the practical implementation of the Journal DAO, we leveraged the Aragon framework to establish a decentralized and transparent infrastructure for academic publishing. The execution of the DAO involved several key steps to ensure a seamless and fair distribution of tokens among participants.
%在期刊 DAO 的实际实施中,我们利用 Aragon 框架建立了一个去中心化和透明的学术出版基础设施。 DAO 的执行涉及几个关键步骤,以确保代币在参与者之间的无缝和公正分配。

\paragraph{Token Distribution Mechanism}
%代币分配机制:

\begin{enumerate}
  \item Author Rewards:
  %作者奖励:

  Authors receive tokens based on the evaluation provided by reviewers during the submission process. The more constructive and impactful the reviews, the higher the token allocation to the authors.
  %作者根据评审人在提交过程中提供的评估获得代币。评审越有建设性和有影响力,作者获得的代币就越多。

  \item Reviewer Incentives:
  %评审者激励:
  
  Reviewers are rewarded with tokens for their valuable contribution to the peer-review process. This includes providing insightful feedback and assisting in maintaining the quality of published work.
  %评审者因其对同行评审过程的宝贵贡献而获得代币。这包括提供有深度的反馈和帮助维护已发布作品的质量。
  
  \item Publication and Download Rewards:
  %出版和下载奖励:
  
  Upon successful publication, both authors and users who download the papers are granted tokens. This encourages not only the creation of quality content but also its dissemination and accessibility.
  %在成功发布后,作者和下载论文的用户都会获得代币。这不仅鼓励了高质量内容的创作,还促进了其传播和获取。

  \item Citation Bonuses:
  %引用奖金:

  Authors receive additional tokens when their published work is cited by other researchers. This incentivizes the production of influential and impactful research that contributes to the academic community.
  %当其他研究人员引用作者已发布的作品时,作者将额外获得代币。这鼓励了产生有影响力和影响力研究的创作。

\end{enumerate}


\paragraph{NFT-Based Finance Distribution}
%基于 NFT 的财务分配:

The finance generated within the Journal DAO is distributed based on the NFT model, where each token holder is entitled to a proportional share. This innovative approach ensures that the financial rewards align with the level of contribution and engagement at various stages of the academic process.
%期刊 DAO 内生成的财务基于 NFT 模型进行分配,其中每个代币持有者有权获得相应份额。这种创新方法确保了财务奖励与在学术流程的各个阶段的贡献和参与程度相一致。

\paragraph{Decentralized Governance}
%去中心化治理:

The DAO operates on a decentralized governance model, allowing token holders to participate in decision-making processes. This ensures that the community has a say in the evolution of the platform, creating a democratic and inclusive environment.
%DAO 采用去中心化治理模型,允许代币持有者参与决策过程。这确保了社区能够参与平台发展,创造了一个民主和包容的环境。

\paragraph{Results and Implications}
%结果和影响:

The implementation of the Journal DAO has yielded positive results in terms of increased engagement, quality submissions, and a more inclusive academic ecosystem. The transparent and automated token distribution mechanisms have effectively addressed issues of ownership and reward distribution, fostering a collaborative and fair scholarly environment.
%期刊 DAO 的实施在提高参与度、质量投稿以及创建更具包容性的学术生态方面取得了积极成果。透明和自动化的代币分配机制有效解决了所有权和奖励分配的问题,促进了合作和公正的学术环境。

In conclusion, the detailed execution of the Journal DAO demonstrates the viability of blockchain and DAO principles in reshaping academic publishing. The emphasis on decentralized governance, token incentives, and transparent finance distribution has the potential to revolutionize the scholarly landscape, making it more accessible, collaborative, and equitable for all participants.
%总的来说,期刊 DAO 的详细执行展示了区块链和 DAO 原则在重塑学术出版方面的可行性。对去中心化治理、代币激励和透明财务分配的强调有望彻底改变学术领域,使其对所有参与者更具可访问性、合作性和公平性。



\begin{figure}[h]
  \centering
  \includegraphics[width=3.2in]{assets/daopaper.png}
  \caption{Distribute Token by DAO}
  \label{fig:distributetoken}
\end{figure}

Figure \ref{fig:distributetoken} is the process:
%以下是该过程的详细步骤:

\begin{enumerate}
  \item Author Submission: The author submits their paper to the DAO-powered academic publishing platform.
  %作者将论文提交给DAO驱动的学术出版平台。
  \item Token Allocation: Tokens are allocated to the author and the publisher. Tokens can be utilized within the DAO ecosystem. 
  %代币分配: 作者和出版商分配代币。代币可以在DAO生态系统中使用。
  \item Community Governance: DAO members, including authors and publishers, may participate in governance decisions related to the platform. 
  %社区治理: DAO成员,包括作者和出版商,可以参与与平台相关的治理决策。
\end{enumerate}


\begin{figure}[h]
  \centering
  \includegraphics[width=3.2in]{assets/download1.png}
  \caption{Distribute Token while User Download}
  \label{fig:financetoken}
\end{figure}

Figure \ref{fig:financetoken} is the process: When a user creates a financial activity, such as paid downloads, rewards are distributed to all holders based on their previous token holdings. Additionally, the user who creating the financial activity receives a certain amount of tokens as a reward for their contribution, becoming a new holder.
%当用户创造了一项财务活动(例如付费下载),根据之前的代币持有情况,将奖励分配给所有持有者。同时,这个用户也会因为其贡献而获得一定数量的代币,成为新的持有者。


\begin{figure}[h]
  \centering
  \includegraphics[width=3.2in]{assets/donwload2.png}
  \caption{Distribute Token while Another User Download}
  \label{fig:financetoken2}
\end{figure}

Figure \ref{fig:financetoken2} is the process when another user makes a paid download, the finance generated will be distributed among the author, publisher, and user1 based on their token holdings. Subsequently, user2 will also become a new token holder as a result of their contribution.
%当另外一个用户付费下载,所生成的财务将根据其代币持有量分配给作者、发行商和用户1。随后,用户2也将因其贡献而成为新的代币持有者。

After voluntarily making a payment, users' ability to participate contributes to a robust incentive mechanism, fostering a sense of autonomy within the entire framework. Through this design, users become direct contributors to financial activities, injecting new value into the framework and creating potential opportunities for self-reward. This decentralized autonomous model empowers users to engage directly in decision-making and contributions, shaping a more open, fair, and virtuous ecosystem. Overall, this autonomous framework cultivates a more positive and sustainable participation experience for users and the entire community.
%用户自主付费后,其参与框架的能力构成了一种积极的激励机制,为整个系统的自治性提供了有力支持。通过这一设计,用户成为财务活动的直接贡献者,其行为不仅为框架注入了新的价值,同时也为自身创造了潜在的奖励机会。这种去中心化的自治模式使得用户能够更加直接地参与决策和贡献,从而塑造了一个更加开放、公正、而且具有良性循环的生态系统。整体而言,这种自治的框架为用户和整个社区创造了更为积极和可持续的参与体验。


\begin{table*}[h!]
  \begin{center}
    \caption{Finance for Token.}
    \label{tab:finance}
    \begin{tabular}{r|r|r|r} % <-- Alignments: 1st column left, 2nd middle and 3rd right, with vertical lines in between
      \textbf{Download Count} & \textbf{Author} & \textbf{Reviewer} & \textbf{First User of Download}\\
      \hline
      1 & 0.4995004995004995 & 0.4995004995004995 & 0.000999000999000999\\
      2 & 0.5039525691699605 & 0.49407114624505927 & 0.001976284584980237\\
      3 & 0.5083088954056696 & 0.488758553274682327 & 0.002932551319648094\\
      ... & ... & ... & ...\\
      10 & 0.5363636363636364 & 0.45454545454545453 & 0.00909090909090909\\
      ... & ... & ... & ...\\
      100 & 0.7129186602870813 & 0.23923444976076555 & 0.04784688995215311\\
      ... & ... & ... & ...\\
    \end{tabular}
  \end{center}
\end{table*}

As the Table \ref{tab:finance} show. Through our detailed simulations and analyses, the incentive mechanisms within the DAO framework emerge as crucial drivers in shaping the dynamics of authorship and user participation. As downloads and citations increase, the token-driven rewards become a powerful motivator for authors, leading to an accumulation of influence and financial gains. This incentive structure not only acknowledges and rewards the contributions of authors but also establishes a direct correlation between their efforts and the benefits they accrue.
%通过我们详尽的模拟和分析,DAO框架内的激励机制成为塑造作者和用户参与动态的关键推动因素。随着下载和引用的增加,基于代币的奖励变得越来越成为作者的强大动力,导致权重和财务收益的积累。这种激励结构不仅承认并奖励了作者的贡献,还在作者的努力和他们获得的收益之间建立了直接的关系。

Furthermore, users who engage with the system by downloading papers witness a direct impact on their influence and, subsequently, on their earnings. This creates a dual incentive structure, where authors and users are mutually motivated to contribute to and participate in the DAO environment. The concept of decentralized autonomy becomes evident as the system operates independently, fostering a self-sustaining loop of contributions, rewards, and governance.
%此外,通过下载论文积极参与系统的用户在他们的影响力和随之而来的收入上见证了直接的影响。这创建了一个双重的激励结构,使作者和用户相互激励,以在DAO环境中做出贡献和参与。去中心化自治的概念在系统自主运作中变得明显,形成了一个贡献、奖励和治理的自我维持循环。
In this context, the DAO framework provides a powerful tool for aligning interests and promoting a fair distribution of rewards based on tangible contributions. The transparency and automation inherent in DAO contribute to a governance model that minimizes external intervention, allowing the ecosystem to evolve organically through the collective actions of its participants. This synergy of incentives and autonomy within DAO not only enhances the overall efficiency of the academic publishing model but also creates a robust and self-regulating environment for authors and users alike.
%在这种情况下,DAO框架为协调利益和促进基于实际贡献的公平奖励分配提供了强大的工具。DAO内在的透明度和自动化有助于最小化外部干预的治理模型,通过参与者的集体行动,使生态系统有机地演变。DAO内的激励和自治的协同作用不仅提升了学术出版模型的整体效率,而且为作者和用户创造了一个强大而自我调节的环境。

Once the paper is on the blockchain, it unequivocally belongs to the author, author is the real owner,that creating endless possibilities, especially in terms of financial activities. This means that the author not only owns their work but can also leverage blockchain technology to create various financial opportunities. Authors can receive rewards through financial activities, which may include paid downloads, knowledge exchanges, collaborative projects, and more. This decentralized framework provides authors with greater creative freedom and potential economic returns, enabling them to be more independent and influential in the academic domain. Overall, putting a paper on the blockchain opens up a new and forward-thinking path for authors.
%论文一旦上链,将完全归属于作者,作者就是真正的拥有者,为作者创造了无限的可能性,特别是在财务活动方面。这意味着作者不仅拥有其作品的所有权,而且可以利用区块链技术创造各种财务机会。作者可以通过财务活动获得回报,这可能包括付费下载、知识交流、合作项目等。这种去中心化的框架为作者提供了更大的创作自由和潜在的经济回报,使其在学术领域更加独立和有影响力。整体而言,论文上链为作者开辟了一个全新的、具有前瞻性的创作道路。



\section{Conclusion}
This paper extensively explores the framework of DAO and provides a thorough analysis of its potential applications in the academic publishing domain. By placing papers on the blockchain, we have achieved transparency in ownership, allowing authors to have complete control over their works while also creating diverse financial opportunities. The autonomous nature of DAO enables users to directly participate in decision-making and contributions, constructing an ecosystem that is open, fair, and characterized by positive feedback loops.
%本文深入探讨了去中心化自治组织(DAO)的框架,并就其在学术出版领域的潜在应用进行了详尽分析。通过将论文上链,我们实现了论文所有权的透明化,使作者能够完全拥有其作品,同时也为其创造了丰富的财务机会。DAO的自治性质使得用户能够直接参与决策和贡献,构建了一个开放、公正、具有良性循环的生态系统。

Under this framework, users can not only pay for paper downloads but also receive rewards through participation in financial activities. This novel academic publishing model grants authors greater creative freedom while motivating users to actively engage, contribute, and share knowledge. The decentralized autonomous design brings a more open and fair publishing mechanism to academia, breaking away from the limitations of traditional academic publishing.
%在这一框架下,用户不仅可以付费下载论文,还可以通过参与财务活动获得回报。这种新型的学术出版模式赋予了作者更大的创作自由,同时也激励了用户积极参与、贡献和分享知识。去中心化自治的设计为学术界带来了更加开放和公正的出版机制,突破了传统学术出版的限制。

In summary, Decentralized Autonomous Organizations inject new vitality into academic publishing, creating a more equitable environment for both authors and readers. This innovative model holds promise for paving new paths in the development of academia, fostering the free dissemination and sharing of knowledge.
%综上所述,去中心化自治组织为学术出版注入了新的活力,为作者和用户创造了更加公平和有利的环境。这一创新模式有望为学术界的发展开辟新的道路,促进知识的自由传播和共享。

% \section*{Acknowledgments}
% This should be a simple paragraph before the References to thank those individuals and institutions who have supported your work on this article.



% {\appendix[Proof of the Zonklar Equations]
% Use $\backslash${\tt{appendix}} if you have a single appendix:
% Do not use $\backslash${\tt{section}} anymore after $\backslash${\tt{appendix}}, only $\backslash${\tt{section*}}.
% If you have multiple appendixes use $\backslash${\tt{appendices}} then use $\backslash${\tt{section}} to start each appendix.
% You must declare a $\backslash${\tt{section}} before using any $\backslash${\tt{subsection}} or using $\backslash${\tt{label}} ($\backslash${\tt{appendices}} by itself
%  starts a section numbered zero.)}



%{\appendices
%\section*{Proof of the First Zonklar Equation}
%Appendix one text goes here.
% You can choose not to have a title for an appendix if you want by leaving the argument blank
%\section*{Proof of the Second Zonklar Equation}
%Appendix two text goes here.}




\bibliography{refs}
\bibliographystyle{IEEEtran}


\newpage

\section{Biography Section}
If you have an EPS/PDF photo (graphicx package needed), extra braces are
 needed around the contents of the optional argument to biography to prevent
 the LaTeX parser from getting confused when it sees the complicated
 $\backslash${\tt{includegraphics}} command within an optional argument. (You can create
 your own custom macro containing the $\backslash${\tt{includegraphics}} command to make things
 simpler here.)
 
\vspace{11pt}

\bf{If you include a photo:}\vspace{-33pt}
\begin{IEEEbiography}[{\includegraphics[width=1in,height=1.25in,clip,keepaspectratio]{fig1}}]{Michael Shell}
Use $\backslash${\tt{begin\{IEEEbiography\}}} and then for the 1st argument use $\backslash${\tt{includegraphics}} to declare and link the author photo.
Use the author name as the 3rd argument followed by the biography text.
\end{IEEEbiography}

\vspace{11pt}

\bf{If you will not include a photo:}\vspace{-33pt}
\begin{IEEEbiographynophoto}{John Doe}
Use $\backslash${\tt{begin\{IEEEbiographynophoto\}}} and the author name as the argument followed by the biography text.
\end{IEEEbiographynophoto}




\vfill

\end{document}


